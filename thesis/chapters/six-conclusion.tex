\mchapter{نتیجه گیری و کار های آینده}

\section{نتیجه گیری}
در این پایان‌نامه، به بررسی و توسعه یک مدل هوشمند برای تشخیص بوی کد با استفاده از مدل‌های زبانی وسیع پرداخته شد. بوی کد به عنوان یک مفهوم کلیدی در مهندسی نرم‌افزار، نشان‌دهنده مشکلات پنهانی در کدهای برنامه‌نویسی است که می‌تواند منجر به کاهش کیفیت، افزایش پیچیدگی در نگهداری و توسعه نرم‌افزار، و ایجاد مشکلات عملکردی در آینده شود. هدف اصلی این پژوهش ارائه روشی نوین برای شناسایی این مشکلات با بهره‌گیری از توانایی‌های مدل‌های زبانی وسیع و روش‌های یادگیری عمیق بود.
\\
نتایج این پژوهش نشان داد که استفاده از مدل‌های زبانی وسیع، به‌ویژه مدل‌هایی که با داده‌های گسترده و متنوع آموزش دیده‌اند، می‌تواند به طور چشمگیری دقت و کارایی در تشخیص بوی کد را بهبود بخشد. این مدل‌ها قادر به تشخیص ۲۸ نوع مختلف از بوی کد بودند و در مقایسه با روش‌های سنتی، نتایج بهتری را ارائه دادند. همچنین، ارزیابی‌های انجام شده بر روی مدل پیشنهادی نشان داد که این روش می‌تواند به عنوان ابزاری مؤثر و کارآمد برای توسعه‌دهندگان نرم‌افزار به کار گرفته شود، و به آنها در بهبود کیفیت کد و کاهش هزینه‌های نگهداری کمک کند.
\\
در عین حال، این پژوهش به چالش‌های موجود در آموزش و بهینه‌سازی مدل‌های زبانی وسیع نیز پرداخت. از جمله این چالش‌ها می‌توان به نیاز به منابع محاسباتی بالا و پیچیدگی‌های مرتبط با تنظیم دقیق مدل‌ها اشاره کرد. با این وجود، راهکارهای ارائه شده در این پایان‌نامه، نظیر استفاده از روش‌های بهینه‌سازی و فشرده‌سازی، می‌تواند به کاهش این چالش‌ها کمک کرده و امکان استفاده از مدل‌های زبانی وسیع را در محیط‌های محدودتر فراهم سازد.
\\
در نهایت، این پژوهش نشان داد که ترکیب مدل‌های زبانی وسیع با دانش مهندسی نرم‌افزار می‌تواند به طور قابل توجهی به بهبود فرآیند توسعه نرم‌افزار و کاهش خطاها و مشکلات کد منجر شود. امید است که نتایج این تحقیق بتواند زمینه‌ساز تحقیقات و توسعه‌های بیشتر در این حوزه گردد و به ارتقاء کیفیت و کارایی نرم‌افزارها کمک کند. همچنین، به پژوهشگران و توسعه‌دهندگان توصیه می‌شود که با ادامه تحقیقات در این زمینه و بررسی روش‌های نوین، به بهبود و تکامل ابزارهای تشخیص بوی کد بپردازند تا بتوانند نرم‌افزارهایی پایدارتر، قابل نگهداری‌تر و با کیفیت‌تر ارائه دهند.

\section{کار های آینده}

در سال‌های اخیر، ادغام روش‌های یادگیری ماشین در حوزه مهندسی نرم‌افزار پیشرفت‌های قابل توجهی داشته است، به ویژه در ارزیابی کیفیت کد. یکی از حوزه‌های مهم تمرکز، تشخیص بوی بد کد است.نشانه‌هایی از مشکلات بالقوه در کد که ممکن است به اشکالات یا دشواری‌های نگهداری منجر شوند. اگرچه مدل‌های کنونی نویدبخش هستند، اما هنوز زمینه‌های زیادی برای بهبود وجود دارد. در ادامه این پروژه به سه حوزه کلیدی زیر اشاره کرد:

\subsection{گسترش مجموعه داده‌ها}
یک عنصر اساسی در هر پروژه یادگیری ماشین، مجموعه داده است. تنوع و حجم داده‌ها به طور مستقیم بر توانایی مدل در تعمیم به سناریوهای مختلف تأثیر می‌گذارد. مجموعه داده‌های فعلی برای تشخیص بوی بد کد، هرچند مؤثر هستند، اما اغلب از گستردگی لازم برای پوشش کامل تنوع سبک‌های کدنویسی، زبان‌ها و محیط‌های مختلف برخوردار نیستند. گسترش مجموعه داده‌ها به منظور شامل کردن طیف وسیع‌تری از زبان‌های برنامه‌نویسی، چارچوب‌های کدنویسی بسیار حیاتی خواهد بود. این گسترش می‌تواند شامل جمع‌آوری پروژه‌های متن‌باز\LTRfootnote{Open Source} بیشتر، ادغام کد از حوزه‌های مختلف و حتی تولید مثال‌های مصنوعی از بوی بد کد باشد. با تنوع‌بخشی به مجموعه داده، مدل می‌تواند برای شناسایی طیف گسترده‌تری از بوهای بد آموزش ببیند و در نتیجه دقت و استحکام خود را در زمینه‌های مختلف بهبود بخشد.

\subsection{بهینه‌سازی مدل}

پس از گسترش مجموعه داده، گام بعدی بهینه‌سازی خود مدل است. اگرچه مدل‌های کنونی در تشخیص بوی بد کد عملکرد مناسبی دارند، ولی همیشه فضایی برای بهبود وجود دارد. کارهای آینده می‌توانند شامل بررسی معماری‌های جایگزین یادگیری ماشین، مانند مدل‌های مبتنی بر مبدل یا شبکه‌های عصبی گرافی باشد که ممکن است با توجه به ساختار کد، عملکرد بهتری داشته باشند. علاوه بر این، روش‌های تنظیم دقیق مانند یادگیری انتقالی، که در آن یک مدل پیش‌آموزش‌دیده بر روی یک مجموعه داده بزرگ و عمومی، به صورت خاص‌تر بر روی یک مجموعه داده کوچک‌تر آموزش داده می‌شود، می‌تواند مفید باشد. تنظیم ابر متغیرها\LTRfootnote{hyper-parameter} و ترکیب مدل‌های مختلف نیز ممکن است بهبود عملکرد را به همراه داشته باشد. هدف از این بهینه‌سازی‌ها افزایش هر دو معیار دقت و یادآوری تشخیص بوی بد کد است که کاهش مثبت‌های کاذب و منفی‌های کاذب برای پذیرش عملی بسیار مهم است.

\subsection{تشخیص در زمان واقعی}

هدف نهایی ابزارهای تشخیص بوی بد کد، یکپارچه‌سازی بدون مشکل در جریان کاری توسعه‌دهنده است، به نحوی که بازخورد فوری ارائه دهند و به حفظ کیفیت بالای کد از همان ابتدا کمک کنند. توسعه یک سیستم تشخیص در زمان واقعی، مانند یک افزونه برای ویرایشگرهای کد محبوب مانند \lr{Visual Studio Code}، \lr{IntelliJ IDEA} یا \lr{Eclipse}، یک مسیر هیجان‌انگیز برای کارهای آینده است. این افزونه می‌تواند کد را در حین نوشتن تحلیل کند، بوی بد کد احتمالی را بلافاصله پرچم‌گذاری کند و پیشنهاداتی برای بهبود ارائه دهد. دستیابی به عملکرد در زمان واقعی نیازمند بهینه‌سازی مدل برای سرعت بدون از دست دادن دقت است و همچنین اطمینان از اینکه افزونه بهینه باشد. چنین ابزاری نه تنها بهره‌وری توسعه‌دهنده را افزایش می‌دهد، بلکه با شناسایی مشکلات در اوایل چرخه توسعه، به نگهداری طولانی‌مدت پروژه‌های نرم‌افزاری کمک می‌کند.
\\
آینده تشخیص بوی بد کد در گسترش مجموعه داده‌ها، بهینه‌سازی مدل‌ها و قابلیت‌های تشخیص در زمان واقعی نهفته است. هر یک از این حوزه‌ها چالش‌های منحصر به فردی را به همراه دارد، اما همچنین توانایی قابل توجهی برای بهبود نحوه تشخیص و رسیدگی به مسائل کیفیت کد ارائه می‌دهد. با تمرکز بر این جنبه‌ها، می‌توانیم به توسعه ابزارهایی نزدیک شویم که نه تنها دقیق و کارآمد هستند، بلکه به طور یکپارچه در فرآیند توسعه نرم‌افزار ادغام می‌شوند و در نهایت به کدهایی با کیفیت بالاتر و قابل نگهداری‌تر منجر می‌شوند.
\clearpage
