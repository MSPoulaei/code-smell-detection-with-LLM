\mchapter{مقدمه}
\pagenumbering{arabic}

تشخیص بوی کد یکی از مفاهیم اساسی در صنعت نرم‌افزار است که اهمیت بسیاری دارد. هر پروژه نرم‌افزاری از یک مجموعه بزرگ از کدها تشکیل شده است که ممکن است توسط چندین توسعه‌دهنده نوشته شده باشند. به همین دلیل، اهمیت تشخیص بوی کد به‌ویژه در پروژه‌های بزرگ بسیار چشمگیر است.

بوی کد به مفهومی گفته می‌شود که نشان‌دهنده کیفیت کد و ساختار آن است. یک کد خوب، کدی است که درک آن ساده است، قابل توسعه بوده و از جوانب مختلفی نظیر بهینه بودن، قابلیت خواندن، تست‌پذیری و قابلیت تغییر برخوردار است.

بوی کد نه تنها برای توسعه‌دهندگان فعلی بلکه برای توسعه‌دهندگان جدیدی که احتمالا بعداً به پروژه می‌پیوندند نیز حائز اهمیت است. با تشخیص بوی کد، توسعه‌دهندگان به راحتی می‌توانند رفتار و عملکرد کد را درک کرده، مشکلات را پیدا کرده و بهبودهای لازم را اعمال کنند.

در مجموع، تشخیص بوی کد به توسعه‌دهندگان کمک می‌کند تا کدهای بهتری بنویسند، پروژه‌های بهتری ارائه دهند و در نهایت، هزینه‌های توسعه و نگهداری را کاهش دهند.\cite{whatiscodesmell}
\newpage
\section[لزوم تشخصیص بوی کد در صنعت نرم افزار]{دلایل لزوم تشخیص بوی کد}
تشخیص و رفع بوی کد باید بخشی از فرآیند توسعه نرم‌افزار باشد. این کار کمک می‌کند تا نرم‌افزارهایی پایدارتر، قابل نگهداری‌تر و با کیفیت‌تر ایجاد شوند و در بلندمدت هزینه‌ها و زمان مورد نیاز برای توسعه و نگهداری را کاهش دهد.
\subsection{افزایش قابلیت نگهداری}
کدهای نرم‌افزاری به مرور زمان پیچیده‌تر می‌شوند، به ویژه هنگامی که تغییرات یا ویژگی‌های جدید به آن اضافه می‌شوند. اگر بوی کد در مراحل اولیه تشخیص داده نشود، این پیچیدگی‌ها ممکن است به چنان حدی برسند که نگهداری و توسعه کد بسیار دشوار شود. تشخیص و رفع بوی کد به بهبود ساختار و معماری کد کمک می‌کند، از انباشت پیچیدگی‌های غیرضروری جلوگیری می‌کند و باعث می‌شود نگهداری و توسعه کد در آینده آسان‌تر شود. این مسئله به خصوص برای تیم‌هایی که روی پروژه‌های بزرگ کار می‌کنند یا توسعه‌دهندگانی که به کدهای قدیمی برمی‌گردند، اهمیت بیشتری پیدا می‌کند.
\subsection{بهبود خوانایی و درک کد}
یکی از عوامل کلیدی برای کدی که بتوان آن را به خوبی نگهداری کرد، خوانایی و قابلیت درک آن است. بوی کد اغلب ناشی از کدهای پیچیده، متدهای بزرگ، یا ساختارهای نامنظم است که خواندن و درک آن‌ها دشوار می‌شود. این مشکل زمانی بحرانی‌تر می‌شود که توسعه‌دهندگان جدید به تیم بپیوندند یا اعضای تیم فعلی پس از مدتی به کدی برگردند که دیگر به یاد نمی‌آورند. با تشخیص بوی کد و بهبود طراحی، کد قابل فهم‌تر و منظم‌تر می‌شود. این به تیم کمک می‌کند تا سریع‌تر با کد کار کند و کمتر دچار اشتباهات ناشی از سوء تفاهم در ساختار کد شوند.
\subsection{کاهش خطر خطاهای آینده}
بوی کد معمولاً نشانه‌ای از مشکلات پنهانی است که ممکن است هنوز به صورت باگ آشکار نشده باشند، اما پتانسیل آن را دارند که در آینده مشکل‌ساز شوند. برای مثال، کدهای تکراری ممکن است منجر به مشکلات سازگاری شوند، یا کلاس‌ها و متدهای پیچیده ممکن است منجر به بروز خطاهایی شوند که پیدا کردن و رفع آن‌ها سخت است. تشخیص بوی کد به شناسایی این مشکلات احتمالی کمک می‌کند و رفع آن‌ها باعث می‌شود که در آینده باگ‌ها و خطاهای کمتری در کد مشاهده شود. به عبارت دیگر، پیشگیری از مشکلات به جای رفع آن‌ها پس از بروز، از هزینه و زمان زیادی صرفه‌جویی می‌کند.
\subsection{افزایش بهره‌وری تیم توسعه}
بوی کد می‌تواند باعث شود که اعضای تیم توسعه زمان بیشتری را برای درک و رفع مشکلات موجود در کد صرف کنند، به خصوص زمانی که این مشکلات به مرور زمان پیچیده‌تر و غیرقابل فهم‌تر می‌شوند. با تشخیص و رفع بوی کد، تیم می‌تواند روی توسعه ویژگی‌های جدید تمرکز کند به جای اینکه وقت زیادی را صرف رفع خطاهای ناشی از ساختار بد کد کند. در نتیجه، تیم توسعه می‌تواند با سرعت بیشتری پیشرفت کند و بهره‌وری آن به طور کلی افزایش یابد. علاوه بر این، کدهای تمیز و بهینه تعامل بهتری بین اعضای تیم ایجاد می‌کنند، زیرا ارتباطات و درک بهتر در مورد ساختار و مسئولیت‌های مختلف کد وجود دارد.
\subsection {پشتیبانی از تغییرات و ارتقاء سیستم}
نرم‌افزارها معمولاً در طول زمان نیاز به تغییر، به‌روزرسانی یا ارتقاء دارند تا با نیازهای جدید کاربران، بازار، یا فناوری سازگار شوند. بوی کد می‌تواند این فرایند را دشوارتر و پرخطرتر کند. به عنوان مثال، وابستگی‌های سنگین بین کلاس‌ها و ماژول‌ها ممکن است باعث شود که تغییر در یک بخش از کد، منجر به مشکلات غیرمنتظره در بخش‌های دیگر شود. اما با تشخیص و رفع بوی کد، ساختار کد بهبود می‌یابد و ماژول‌ها به صورت مستقل‌تر عمل می‌کنند. این امر به توسعه‌دهندگان اجازه می‌دهد که تغییرات و ارتقاءهای لازم را بدون ایجاد مشکلات ناخواسته اعمال کنند.
\\
ادامه‌ی پروژه به پنج فصل تقسیم شده است. در فصل ۲ پژوهش‌های مرتبط انجام شده را مرور می‌کنیم. در فصل ۳ به شرح مسئله‌ی توانایی شناسایی و دسته‌بندی بوی کدها می‌پردازیم. در فصل ۴ روش پیشنهادی برای توانایی شناسایی و دسته‌بندی بوی کدها را شرح می‌دهیم. در فصل ۵ نحوه‌ی ارزیابی مدل را ارائه می‌دهیم. در انتها در فصل ۶ یک جمع‌بندی کلی از تمامی مطالب ارائه می‌دهیم و پیشنهاداتی برای گسترش روش پیشنهادی عنوان می‌کنیم.

\clearpage
