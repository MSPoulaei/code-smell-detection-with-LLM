\mchapter{شرح مسئله}
در پروژه حاضر، هدف اصلی توسعه یک سیستم هوشمند است که توانایی شناسایی و دسته‌بندی بوی کدها در نمونه‌های کد را داشته باشد. بوی کدها به نشانه‌هایی در کد اشاره دارند که معمولاً نشان‌دهنده مشکلات طراحی یا پیاده‌سازی هستند و می‌توانند منجر به کاهش کیفیت کد، افزایش پیچیدگی و دشواری در نگهداری و توسعه شوند. شناسایی بوی کدها به توسعه‌دهندگان کمک می‌کند تا پیش از آن که مشکلات عمده‌ای در کد بروز کنند، آنها را شناسایی و رفع کنند.

طبق جدول \ref{tab:code_smells_classes}
در این پروژه، ۲۸ نوع بوی کد مختلف مورد شناسایی قرار می‌گیرند. هر یک از این بوها ویژگی‌ها و نشانه‌های خاص خود را دارند که باید توسط سیستم شناسایی شوند.برخلاف دسته‌بندی تک‌برچسبی که هر نمونه تنها یک برچسب دریافت می‌کند، در دسته‌بندی چندبرچسبی هر نمونه کد می‌تواند چندین برچسب را به صورت همزمان داشته باشد. به عبارت دیگر، هر نمونه کد ممکن است شامل چندین نوع بوی کد باشد و سیستم باید بتواند تمامی بوی کدهای موجود در یک نمونه را شناسایی و برچسب‌گذاری کند.
\clearpage
\begin{longtable}{|>{\centering\arraybackslash}p{5cm}|>{\raggedright\arraybackslash}p{10cm}|}
    \caption{لیست بوی بد کد و توضیحات آن‌ها} 
    \label{tab:code_smells_classes}                                       \\
    \hline
    \textbf{نام بوی بد کد}                       & \textbf{توضیحات}                                               \\
    \hline
    \lr{Missing Hierarchy}                       & نداشتن سلسله‌مراتب مناسب در ساختار کد                           \\
    \hline
    \lr{Long Parameter List}                     & لیست پارامترهای طولانی که درک و استفاده از متد را دشوار می‌کند  \\
    \hline
    \lr{Unnecessary Abstraction}                 & وجود انتزاعاتی که هیچ نیازی به آن‌ها نیست                       \\
    \hline
    \lr{Imperative Abstraction}                  & استفاده از انتزاعاتی که از سبک برنامه‌نویسی دستوری پیروی می‌کنند \\
    \hline
    \lr{Empty Catch Clause}                      & بلوک‌های کچ خالی که خطاها را به درستی مدیریت نمی‌کنند            \\
    \hline
    \lr{Deficient Encapsulation}                 & عدم مخفی‌سازی کافی داده‌ها و متدها در کلاس‌ها                     \\
    \hline
    \lr{Long Identifier}                         & استفاده از شناسه‌های طولانی که خوانایی کد را کاهش می‌دهد         \\
    \hline
    \lr{Multifaceted Abstraction}                & انتزاعاتی که وظایف متعددی را انجام می‌دهند                      \\
    \hline
    \lr{Wide Hierarchy}                          & سلسله‌مراتب گسترده‌ای که مدیریت را پیچیده می‌کند                  \\
    \hline
    \lr{Complex Conditional}                     & شرط‌های پیچیده که کد را سخت‌خوان و دشوار می‌سازد                  \\
    \hline
    \lr{Rebellious Hierarchy}                    & سلسله مراتبی که از اصول طراحی پیروی نمی‌کند                     \\
    \hline
    \lr{Magic Number}                            & استفاده از اعداد جادویی که معنا را در کد مخفی می‌کنند           \\
    \hline
    \lr{Missing default}                         & نداشتن مورد پیش‌فرض در ساختارهای انتخاب                         \\
    \hline
    \lr{Long Method}                             & متدهای طولانی که فهم و نگهداری آن‌ها سخت است                    \\
    \hline
    \lr{Broken Modularization}                   & ماژول‌های شکسته که وظایف را به درستی تقسیم نمی‌کنند              \\
    \hline
    \lr{Broken Hierarchy}                        & سلسله مراتبی که به درستی ساختار نیافته است                     \\
    \hline
    \lr{Unutilized Abstraction}                  & انتزاعاتی که به درستی استفاده نمی‌شوند                          \\
    \hline
    \lr{Long Statement}                          & جملات برنامه‌نویسی طولانی که خوانایی را کاهش می‌دهد              \\
    \hline
    \lr{Cyclic-Dependent Modularization}         & ماژول‌هایی که به صورت چرخه‌ای به هم وابسته‌اند                    \\
    \hline
    \lr{Multipath Hierarchy}                     & سلسله مراتبی که مسیرهای متعددی دارد                            \\
    \hline
    \lr{Deep Hierarchy}                          & سلسله‌مراتب عمیقی که پیگیری آن دشوار است                        \\
    \hline
    \lr{Hub-like Modularization}                 & ماژول‌هایی که به‌طور متمرکز عمل می‌کنند و وابستگی زیادی دارند     \\
    \hline
    \lr{Insufficient Modularization}             & ماژول‌هایی که به اندازه کافی تقسیم نشده‌اند                      \\
    \hline
    \lr{Cyclic Hierarchy}                        & سلسله مراتبی که به صورت چرخه‌ای به هم وابسته‌اند                 \\
    \hline
    \lr{Unexploited Encapsulation}               & کپسوله‌سازی که به درستی بهره‌برداری نشده است                     \\
    \hline
    \lr{Abstract Function Call From Constructor} & فراخوانی توابع انتزاعی از سازنده                               \\
    \hline
    \lr{Complex Method}                          & متدهای پیچیده که درک و نگهداری آن‌ها دشوار است                  \\
    \hline
\end{longtable}

ورودی سیستم شامل مجموعه‌ای از نمونه‌های کد است که می‌تواند از زبان‌های برنامه‌نویسی مختلف باشد. هر نمونه کد باید مورد بررسی قرار گیرد تا مشخص شود که کدام یک از ۲۸ بوی کد در آن وجود دارد.
خروجی سیستم شامل مجموعه‌ای از برچسب‌ها برای هر نمونه کد است که نشان می‌دهد کدام بوی کدها در آن نمونه وجود دارند. این خروجی می‌تواند به توسعه‌دهندگان اطلاعات ارزشمندی برای بهبود کیفیت کد و رفع مشکلات ارائه دهد.
\\
شناسایی بوی کدها به دلیل تنوع و پیچیدگی نشانه‌ها و ویژگی‌های هر بوی کد، چالش‌برانگیز است. برخی از بوها ممکن است به صورت ترکیبی از چندین نشانه ظاهر شوند یا در بخش‌های مختلف کد پراکنده باشند.علاوه بر این، دسته‌بندی چندبرچسبی نیازمند الگوریتم‌ها و مدل‌های پیچیده‌ای است که بتوانند به درستی و با دقت بالا برچسب‌های متعدد را به هر نمونه کد اختصاص دهند.\cite{guggulothu2019codesmelldetectionusing}\cite{guggulothu2020code}
\clearpage