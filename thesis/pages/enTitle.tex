% !TeX root=main.tex
% در این فایل، عنوان پایان‌نامه، مشخصات خود و چکیده پایان‌نامه را به انگلیسی، وارد کنید.

%%%%%%%%%%%%%%%%%%%%%%%%%%%%%%%%%%%%
\baselineskip=.6cm
\begin{latin}
    \latinuniversity{Iran University of Science and Technology}
    \latinfaculty{Computer Engineering Department}
    \latinsubject{Computer Engineering }
    \latinfield{Software Engineering}
    \latintitle{Code Smell Detection with Suggested Large Language Models}
    \firstlatinsupervisor{Dr. Saeid Parsa}
    %\secondlatinsupervisor{Second Supervisor}
    %\firstlatinadvisor{Dr. Einollah Khanjari}
    \secondlatinadvisor{Dr. X}
    \latinnameF{\lr{Mohammad Sadegh}}
    \latinsurnameF{\lr{Poulaei Moziraji}}
    \latinnameS{\lr{Sayin}}
    \latinsurnameS{\lr{Ala}}
    \latinthesisdate{August 2024}
    \latinkeywords{}
    % \latinkeywords{Task Offloading, Edge Computing, Markov Chains, Linear Programming, Cloud Computing}
    \en-abstract{}
    % \en-abstract{Edge computing is a distributed computing paradigm that seeks to provide users with lower response times, lower power consumption, and mobility management by bringing computing resources closer to the network edge. Since its introduction, edge computing and its standard implementations, such as Multi-access Edge Computing, have faced one important challenge: How to design efficient task offloading policies? Furthermore, with the rapid growth of the smartphone and IoT industry, many new types of applications have been introduced to the internet, each having different resource needs. Thus, taking into account the heterogeneity of user tasks becomes an essential factor when designing task offloading policies for edge computing environments. This paper introduces a method for finding the delay-optimal task offloading policy under the power consumption constraint. The method consists of two steps. First, the offloading system is modeled using Discrete-time Markov Chains. Then, an algorithm based on linear programming is used to find the optimal task offloading policy for the created model. In addition to discussing the problem mathematically, we introduce a new software framework, written in the Kotlin language, which allows users to find the optimal task offloading policy for a given system. This framework can also benchmark the optimal policy's effectiveness using simulation.
    % }
    \latinfirstPage
\end{latin}
