% !TeX root=main.tex
% در این فایل، عنوان پایان‌نامه، مشخصات خود، متن تقدیمی‌، ستایش، سپاس‌گزاری و چکیده پایان‌نامه را به فارسی، وارد کنید.
% توجه داشته باشید که جدول حاوی مشخصات پروژه/پایان‌نامه/رساله و همچنین، مشخصات داخل آن، به طور خودکار، درج می‌شود.
%%%%%%%%%%%%%%%%%%%%%%%%%%%%%%%%%%%%
% دانشگاه خود را وارد کنید
\university{علم و صنعت ایران}
% دانشکده، آموزشکده و یا پژوهشکده  خود را وارد کنید
\faculty{دانشکده‌ی مهندسی کامپیوتر}
% گروه آموزشی خود را وارد کنید
\department{گروه نرم افزار}
% گروه آموزشی خود را وارد کنید
\subject{مهندسی کامپیوتر}
% گرایش خود را وارد کنید
% عنوان پایان‌نامه را وارد کنید
\title{تشخیص بوی کد با استفاده از مدل های زبانی وسیع پیشنهادی}
% نام استاد(ان) راهنما را وارد کنید
\firstsupervisor{دکتر سعید پارسا}
%\secondsupervisor{استاد راهنمای دوم}
% نام استاد(دان) مشاور را وارد کنید. چنانچه استاد مشاور ندارید، دستور پایین را غیرفعال کنید.
%\firstadvisor{استاد مشاور اول}
%\secondadvisor{استاد مشاور دوم}
% نام دانشجو را وارد کنید
\nameF{محمدصادق}
% نام خانوادگی دانشجو را وارد کنید
\surnameF{پولائی موزیرجی}
% شماره دانشجویی دانشجو را وارد کنید
\nameS{سایین}
% نام خانوادگی دانشجو را وارد کنید
\surnameS{اعلا}
% شماره دانشجویی دانشجو را وارد کنید
\studentIDF{99521145}
% شماره دانشجویی دانشجو را وارد کنید
\studentIDS{99400023}
% تاریخ پایان‌نامه را وارد کنید
\thesisdate{شهریور ۱۴۰۳}
% به صورت پیش‌فرض برای پایان‌نامه‌های کارشناسی تا دکترا به ترتیب از عبارات «پروژه»، «پایان‌نامه» و »رساله» استفاده می‌شود؛ اگر  نمی‌پسندید هر عنوانی را که مایلید در دستور زیر قرار داده و آنرا از حالت توضیح خارج کنید.
%\projectLabel{پایان‌نامه}

% به صورت پیش‌فرض برای عناوین مقاطع تحصیلی کارشناسی تا دکترا به ترتیب از عبارات «کارشناسی»، «کارشناسی ارشد» و »دکترا» استفاده می‌شود؛ اگر  نمی‌پسندید هر عنوانی را که مایلید در دستور زیر قرار داده و آنرا از حالت توضیح خارج کنید.
\degree{کارشناسی}
\def \CurrentProject {پروژه‌ی فعلی }
\firstPage
\besmPage
\davaranPage

%\vspace{.5cm}
% در این قسمت اسامی اساتید راهنما، مشاور و داور باید به صورت دستی وارد شوند
%\renewcommand{\arraystretch}{1.2}
\begin{center}
	\begin{tabular}{| p{8mm} | p{18mm} | p{.17\textwidth} |p{14mm}|p{.2\textwidth}|c|}
		\hline
		ردیف                               & سمت                       & نام و نام خانوادگی                & مرتبه‌ی \newline دانشگاهی & دانشگاه یا مؤسسه & امضــــــــــــا \\
		\hline
		۱                                  & استاد راهنما              & دکتر \newline سعید پارسا
		                                   & دانشیار                   & دانشگاه \newline علم و صنعت ایران &                                                                \\
		\hline
		۲                                  & استاد داور \newline داخلی & دکتر \newline ....                & .....                    &
		دانشگاه  \newline علم ‌و صنعت ایران &                                                                                                                                \\
		\hline
	\end{tabular}
\end{center}

\esalatPage
\mojavezPage

\newpage

% 
% سپاس‌گزاری
\begin{acknowledgementpage}
	سپاس خداوندگار حکیم را که با لطف بی‌کران خود، آدمی را زیور عقل آراست.


	در آغاز وظیفه‌  خود  می‌دانم از زحمات بی‌دریغ استاد  راهنمای خود،  جناب آقای دکتر سعید پارسا، صمیمانه تشکر و  قدردانی کنم  که قطعاً بدون راهنمایی‌های ارزنده‌  ایشان، این مجموعه  به انجام  نمی‌رسید.

	از جناب  آقای  دکتر سعید پارسا  که زحمت  مطالعه و مشاوره‌  این رساله را تقبل  فرمودند و در آماده سازی  این رساله، به نحو احسن اینجانب را مورد راهنمایی قرار دادند، کمال امتنان را دارم.

	% همچنین لازم می‌دانم از پدید آورندگان بسته زی‌پرشین، مخصوصاً جناب آقای  وفا خلیقی، که این پایان‌نامه با استفاده از این بسته، آماده شده است و همه دوستانمان در گروه پارسی‌لاتک کمال قدردانی را داشته باشم.

	در پایان، بوسه می‌زنم بر دستان خداوندگاران مهر و مهربانی، پدر و مادر عزیزم و بعد از خدا، ستایش می‌کنم وجود مقدس‌شان را و تشکر می‌کنم از خانواده عزیزم به پاس عاطفه سرشار و گرمای امیدبخش وجودشان، که بهترین پشتیبان من بودند.
	% با استفاده از دستور زیر، امضای شما، به طور خودکار، درج می‌شود.
	\signature
\end{acknowledgementpage}
%%%%%%%%%%%%%%%%%%%%%%%%%%%%%%%%%%%%
% کلمات کلیدی پایان‌نامه را وارد کنید
\keywords{بوی کد، مدل‌های زبانی وسیع، یادگیری عمیق، تشخیص خودکار، ترنسفورمر، بهینه‌سازی مدل}

\fa-abstract{در این پژوهش، به بررسی و پیشنهاد مدلی برای تشخیص بوی کد با استفاده از مدل‌های زبانی وسیع پرداخته شده است. بوی کد به مفاهیم و ویژگی‌هایی در کد برنامه‌نویسی اشاره دارد که می‌تواند نشان‌دهنده مشکلات عمیق‌تری در طراحی و پیاده‌سازی نرم‌افزار باشد. این مشکلات ممکن است به کاهش کیفیت کد و افزایش پیچیدگی در نگهداری و توسعه منجر شوند. روش پیشنهادی این پژوهش شامل استفاده از مدل‌های زبانی وسیع است که با استفاده از مجموعه داده‌های برچسب‌خورده آموزش دیده و توانایی تشخیص ۲۸ نوع مختلف از بوی کد را دارد. مدل پیشنهادی با بهره‌گیری از تکنیک‌های یادگیری عمیق و معماری‌های پیشرفته نظیر ترنسفورمرها توسعه یافته است. نتایج ارزیابی‌ها نشان می‌دهد که مدل پیشنهادی قادر است بهبود قابل توجهی در دقت تشخیص بوی کد ایجاد کند و می‌تواند به عنوان ابزاری مؤثر برای توسعه‌دهندگان نرم‌افزار مورد استفاده قرار گیرد. همچنین در این پژوهش به چالش‌های موجود در آموزش و بهینه‌سازی مدل‌های زبانی وسیع پرداخته شده و راهکارهایی برای ارتقای عملکرد مدل ارائه شده است .}

\abstractPage

\newpage\clearpage